\chapter{METODOLOGI PENELITIAN}

\section{Metode Penelitian}

Penelitian ini menggunakan metode Research and Development (R\&D) dengan pendekatan pengembangan perangkat lunak menggunakan model Waterfall yang telah dimodifikasi. Metode ini dipilih karena sesuai dengan karakteristik pengembangan aplikasi web yang memerlukan tahapan sistematis dan terstruktur.

\section{Tahapan Penelitian}

Penelitian ini dilakukan melalui beberapa tahapan sebagai berikut:

\subsection{Analisis Kebutuhan}

Tahap ini meliputi:
\begin{enumerate}
    \item Identifikasi kebutuhan fungsional dan non-fungsional sistem
    \item Analisis dataset URL \textit{phishing} dan legitimate
    \item Studi literatur tentang teknik deteksi \textit{phishing}
    \item Analisis fitur leksikal yang relevan
\end{enumerate}

\subsection{Perancangan Sistem}

Tahap perancangan meliputi:
\begin{enumerate}
    \item Perancangan arsitektur sistem
    \item Perancangan database dan struktur data
    \item Perancangan API endpoint
    \item Perancangan antarmuka pengguna (UI/UX)
    \item Perancangan alur kerja sistem (flowchart)
\end{enumerate}

\subsection{Pengembangan Model Machine Learning}

Tahap ini meliputi:
\begin{enumerate}
    \item Pengumpulan dan preprocessing dataset
    \item Ekstraksi fitur leksikal dari URL
    \item Pembagian data training dan testing (80:20)
    \item Training model menggunakan algoritma Random Forest
    \item Evaluasi dan optimasi model
    \item Serialisasi model untuk deployment
\end{enumerate}

\subsection{Implementasi Aplikasi Web}

Tahap implementasi meliputi:
\begin{enumerate}
    \item Pengembangan backend API menggunakan Django REST Framework
    \item Implementasi endpoint untuk prediksi URL
    \item Integrasi model machine learning dengan backend
    \item Pengembangan frontend menggunakan Next.js
    \item Implementasi antarmuka pengguna
    \item Integrasi frontend dengan backend API
\end{enumerate}

\subsection{Testing dan Evaluasi}

Tahap testing meliputi:
\begin{enumerate}
    \item Unit testing untuk setiap komponen
    \item Integration testing untuk API
    \item User acceptance testing (UAT)
    \item Evaluasi kinerja model (akurasi, presisi, recall, F1-score)
    \item Evaluasi performa aplikasi (response time, throughput)
\end{enumerate}

\subsection{Dokumentasi}

Tahap dokumentasi meliputi:
\begin{enumerate}
    \item Dokumentasi kode program
    \item Dokumentasi API
    \item Dokumentasi pengguna
    \item Penyusunan laporan penelitian
\end{enumerate}

\section{Diagram Alur Sistem}

\subsection{Flowchart Sistem Keseluruhan}

Flowchart sistem keseluruhan menggambarkan alur kerja aplikasi dari input URL hingga output hasil prediksi.

\begin{figure}[htbp]
    \centering
    \includegraphics[width=0.7\textwidth]{diagrams/system-flowchart.png}
    \caption{Flowchart Sistem Keseluruhan}
    \label{fig:system-flowchart}
\end{figure}

\clearpage

\subsection{Use Case Diagram}

Use case diagram menggambarkan interaksi antara pengguna dengan sistem.

\begin{figure}[htbp]
    \centering
    \includegraphics[width=0.8\textwidth]{diagrams/usecase-diagram.png}
    \caption{Use Case Diagram}
    \label{fig:usecase}
\end{figure}

\clearpage

\subsection{Sequence Diagram}

Sequence diagram menggambarkan urutan interaksi antara komponen sistem dalam proses prediksi URL.

\begin{figure}[htbp]
    \centering
    \includegraphics[width=0.95\textwidth]{diagrams/sequence-diagram.png}
    \caption{Sequence Diagram Prediksi URL}
    \label{fig:sequence}
\end{figure}

\clearpage

\subsection{Arsitektur Sistem}

Diagram arsitektur sistem menggambarkan komponen-komponen utama dan hubungan antar komponen.

\begin{figure}[htbp]
    \centering
    \includegraphics[width=0.95\textwidth]{diagrams/architecture-diagram.png}
    \caption{Arsitektur Sistem}
    \label{fig:architecture}
\end{figure}

\clearpage

\section{Alat dan Bahan}

\subsection{Perangkat Keras}

Spesifikasi perangkat keras yang digunakan:
\begin{itemize}
    \item Processor: Intel Core i5 atau setara
    \item RAM: Minimal 8 GB
    \item Storage: Minimal 256 GB SSD
    \item Koneksi Internet: Minimal 10 Mbps
\end{itemize}

\subsection{Perangkat Lunak}

Perangkat lunak yang digunakan:
\begin{itemize}
    \item Sistem Operasi: Windows 10/11, macOS, atau Linux
    \item Python 3.8 atau lebih tinggi
    \item Node.js 18 atau lebih tinggi
    \item Django 4.2
    \item Django REST Framework 3.14
    \item Next.js 14
    \item Scikit-learn 1.6
    \item Visual Studio Code atau IDE lainnya
    \item Git untuk version control
\end{itemize}

\subsection{Dataset}

Dataset yang digunakan:
\begin{itemize}
    \item Nama: Malicious Phish Dataset
    \item Sumber: Kaggle / UCI Machine Learning Repository
    \item Jumlah: Lebih dari 10,000 URL
    \item Kategori: Benign, Phishing, Malware, Defacement
    \item Format: CSV
\end{itemize}

\section{Jadwal Penelitian}

Penelitian ini direncanakan berlangsung selama 4 bulan dengan rincian sebagai berikut:

\begin{table}[h]
\centering
\caption{Jadwal Penelitian}
\label{tab:jadwal}
\begin{tabular}{|l|c|c|c|c|}
\hline
\textbf{Kegiatan} & \textbf{Bulan 1} & \textbf{Bulan 2} & \textbf{Bulan 3} & \textbf{Bulan 4} \\
\hline
Analisis Kebutuhan & X & & & \\
\hline
Perancangan Sistem & X & X & & \\
\hline
Pengembangan Model ML & & X & X & \\
\hline
Implementasi Backend & & & X & \\
\hline
Implementasi Frontend & & & X & X \\
\hline
Testing \& Evaluasi & & & & X \\
\hline
Dokumentasi & X & X & X & X \\
\hline
\end{tabular}
\end{table}
